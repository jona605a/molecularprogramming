\documentclass[11pt,a4paper]{article}
\usepackage[margin=1in]{geometry}% Tweak page margins
\usepackage{graphicx} % Required for inserting images
\usepackage{float} % Positioning of figures
\usepackage{xcolor}
\usepackage{tikz-cd}
%\usepackage{tikz-cd}
\usepackage[most]{tcolorbox}
\usepackage[outputdir=../]{minted}
\usepackage{biblatex}
%\bibliography{ref}
\usepackage{hyperref}
\usepackage{verbatim}
\usepackage{forest}



\definecolor{folderbg}{RGB}{124,166,198}
\definecolor{folderborder}{RGB}{110,144,169}

\def\Size{4pt}
\tikzset{
  folder/.pic={
    \filldraw[draw=folderborder,top color=folderbg!50,bottom color=folderbg]
      (-1.05*\Size,0.2\Size+5pt) rectangle ++(.75*\Size,-0.2\Size-5pt);  
    \filldraw[draw=folderborder,top color=folderbg!50,bottom color=folderbg]
      (-1.15*\Size,-\Size) rectangle (1.15*\Size,\Size);
  }
}


\title{Molecular programming with \textit{CRN++}, a 02257 project}
\author{
    Hans Henrik Hermansen s194042\\
    Jonathan S. Højlev s194684\\
    Rani Ey. í Bø s194067}
\date{26/06 2024}

\begin{document}
\pagenumbering{gobble}

\maketitle
\section{Abstract}
This report explores our implementation of the CRN++ programming language as outlined by Vasic, Soloveichik, and Khurshid. Our primary aim was to deepen our proficiency with the F\# programming language by translating CRN++ into functional software components. We improved the grammar and the type checker, parser and more. Additionally, we made the solution to the ODE more efficiently computable, by integrating matrix operations. These optimizations made the simulations much faster, which among other things enabled us to introduce more tests. Speaking of tests, our work also included many tests that aim to verify the functionality of the code. Furthermore, we developed visualization tools to graphically depict the dynamic changes in chemical concentrations over time.\\

Through this project, we not only enhanced our technical skills in F\# but also gained insight into the cool world of chemical computation. It was cool to learn how chemistry is Turing complete and to us the functional paradigm to implement a, for us, totally novel, fully parallel, chemical programming paradigm.




\newpage

\pagenumbering{arabic}

\section{An interpreter for CRN++}

In the article \textit{CRN++: Molecular Programming Language}, Marko Vasic, David Soloveichik, and Sarfraz Khurshid define a language for programming chemical reactions to perform computation. They define a context-free grammar for this language and explain some of the limitations of the uses and expressiveness of the language when real-world chemical reactions enter the equation. 


\subsection{Grammar}

Looking at the grammar defined by the article in their listing 1.1, there are immediately some changes we would like to make. Besides this, we impose a few syntactic restrictions to ensure all parsed programs behave as expected. 

In the following, our changes to the grammar is listed. The complete revised grammar can be seen in appendix \ref{sec:grammar_revised}. 

In the article's grammar, it is possible to interchange \texttt{'conc'} and \texttt{'step'} rules, which means a \texttt{'conc'} rule in the bottom of a program could set the concentration of a species used in the beginning. We deemed this unintuitive and in our grammar, all \texttt{'conc'} rules must come before all \texttt{'step'} rules. This was done by changing the beginning of the program to the following:
\begin{tabbing}
    $\langle \text{Crn} \rangle$ \,::=\; \= \texttt{'crn=\{'$\langle \text{ConcRootSList} \rangle$'\}'} \\
    
    $\langle \text{ConcRootSList} \rangle$ \,::=\;  $\langle \text{ConcS} \rangle$ ',' $\langle \text{ConcRootList} \rangle$ \\
    \>\textbar \, $\langle \text{RootSList} \rangle$ \\

    $\langle \text{RootSList} \rangle$ \,::=\;  $\langle \text{StepS} \rangle$ \\
    \>\textbar \, $\langle \text{StepS} \rangle$ ',' $\langle \text{RootSList} \rangle$
\end{tabbing}
With these rules, a program can have zero or more \texttt{'conc'} statements and at least one \texttt{'step'}. 

In the article's grammar, \texttt{CommandS} includes \texttt{ArithmeticS} nad \texttt{CmpS} but they are not defined. We assume that they refer to the defined but not mentioned \texttt{ModuleS}, since \texttt{ModuleS} contains arithmetic expressions and includes the \texttt{'cmp'} rule. When revising the grammar, we decided to define \texttt{CommandS} as simply each of these, along with \texttt{ConditionalS} which was already a part of it. The result can be seen in appendix \ref{sec:grammar_revised}. 

Other than \texttt{ConcS}, there are also other restrictions, all of which we found easier to implement using syntactic restrictions. Firstly, all numbers must be non-negative, since chemical concentrations are always positive. Furthermore, to ensure fast and deterministic convergence of all ODEs, there can not be any cycles where one species depends on itself, nor can a species be written to twice in one step. Notice that this implicitly disallows multiple \texttt{cmp} in a single step since all \texttt{cmp} write to the same flags. Lastly, we enforce that all \texttt{ConditionalS} in a step must be mutually exclusive, e.g. \texttt{ifLT} cannot be used together with \texttt{ifLE} since both can be true simultaneously. Additionally, there can not be any nested \texttt{if}. This does not reduce expressiveness of the language and makes it simpler to check the restrictions.


\subsection{Abstract syntax tree (AST)}

The AST is rooted in Root: 
\begin{minted}{haskell}
    type Root = R of Conc List * Step List
\end{minted}
Which forces all the \texttt{Concs} to come before the steps, where Conc is defined as:
\begin{minted}{haskell}
    type Conc = C of species * float
\end{minted}
This is then followed by all the \texttt{Steps}:
\begin{minted}{haskell}
    type Step = S of Command List
\end{minted}
Where the definition of \texttt{Command} is:
\begin{minted}{haskell}
    type Command =
        | Ld of species * species
        | Add of species * species * species
        | Sub of species * species * species
        | Mul of species * species * species
        | Div of species * species * species
        | Sqrt of species * species
        | Cmp of species * species
        | Rx of species List * species List * float
        | IfGT of Command List
        | IfGE of Command List
        | IfEQ of Command List
        | IfLT of Command List
        | IfLE of Command List
\end{minted}
And \texttt{species} is just a string, that signifies the name of the species:
\begin{minted}{haskell}
    type species = string
\end{minted}
The tree representation of the ..... program can be seen below:
\subsection{Parser}


\subsection{Type checker}
After parsing, we have an AST, however, this AST might not follow all the rules extraneous to the grammar. By recursively matching on each \texttt{StepList} in the AST, we check all the constraints mentioned in the section on the grammar, for example, that the conditionals in a step do not overlap and that no species is written to twice. While traversing the \texttt{StepList}, a graph $G$ is built, whose nodes represent species and an edge from $a\to b$ signals that $a$ is input to $b$. By then trying to topologically sort $G$, we can quickly check if $G$ is a $DAG$ and thereby does not contain any cycles.

\subsection{State and interpreter}
The state is quite simply just a map from each species to its corresponding value:
\begin{minted}{haskell}
    type State = Map<species, float>
\end{minted}
Since each \texttt{State} only depends on the last \texttt{State} and the steps, one can make an infinite \texttt{State} sequence, by implementing a function that generates the next state based on the current state:
\begin{minted}{haskell}
    let doStep (S(cl): Step) (state: State) : State = doCommandList cl state
\end{minted}
and utilizing it in an unfold. The interoperate function then becomes:
\begin{minted}{haskell}
    let interpretProgram (R(concl, stepl)) =

        if not (isTyped (R(concl, stepl))) then
            failwith "Does not typecheck"
        else
            let initialState = getInitialState concl
    
            Seq.unfold
                (fun (state, i) ->
                    let nextState = doStep (List.item (i % List.length stepl) stepl)
                        state Some(nextState, (nextState, i + 1)))
                (initialState, 0)
\end{minted}

Where \texttt{getInitialState} just reads all the initial concentrations:
\begin{minted}{haskell}
    let getInitialState (concl: Conc List) : State =
        List.fold (fun state (C(s, c)) -> Map.add s c state) Map.empty concl
\end{minted}

\subsection{Testing}
We wanted to use property based testing, in the hope that it would be as applicable here as it was when drawing trees in the first project. However, there were some obstacles. It was not easy to get \texttt{FsCheck} to generate an AST that could have resulted from a valid parsed program. Furthermore, generating programs and then parsing them was also not doable, since the fraction of valid programs over invalid programs is vanishingly small. Thus we opted for a more traditional approach to testing, where we have programs that we know are correct and programs that we know are flawed, and we make tests on these. \texttt{FsCheck} then picks one of these programs and we check that mapping shuffle on each \texttt{StepList} does not change whether it type checks or the resulting sequence of states. Since one cannot compare infinite sequences directly, we only check whether the first 50 steps match.\\

To increase the coverage of the tests, the individual arithmetic operators were tested in isolation, where \texttt{FsCheck} picks for example the numbers to add. These were then compared to the result from the arithmetic operations in \texttt{F\#}.

\subsection{Visualization}
To visualize the sequence of states, we first output the sequence to a file, in a manner where we can easily read it into Python.

We then load this into Python and get a list of all the species, and a matrix where row $i$ corresponds to the concentration in step $i$. The $j$th index in all rows refers to the concentration of the $j$th species, therefore, by transposing the matrix, the $i$th row becomes the data points that are to be plotted for species $i$. By then having two copies of each value except the last value, and having the x-axis $0,0.99,1,1.99,2,2.99\dots$ instead of $0,1,2\dots$ we get the stair plot that clearly shows the step-wise process instead of a continuous progression between steps.

To make the plot a bit nicer and easier to read, the lines have different opaque colours and one can elect which species are plotted through command line arguments. An example is shown in the following figure:
\begin{figure}[H]
    \centering
    \includegraphics[scale=0.4]{report/figures/examplePlot.png}
    %\caption{Caption}
    %\label{fig:enter-label}
\end{figure}

\subsection{Assessment}
\section{Chemical reactions}

\subsection{Reactions and CRN}
A reaction is just a set of reactants, a set of products, and a number $\kappa$ representing the speed of the reaction. Each reactant and product is some quantity of a species, these can be modelled as:  
\begin{minted}{haskell}
    type Reaction = Rxn of Map<species, float> * Map<species, float> * float
\end{minted}
Where the maps map a species to its concentration, the first map is the reactants, the second map is the products and the final float is $\kappa$.\\

Then \texttt{CRN} can be defined as the union of all the reactions, which can be stored in a reaction list:
\begin{minted}{haskell}
    type CRN = Reaction List
\end{minted}


\subsection{Simulation}
The ODE that we want to simulate describes the rate of change of concentration for a species \( S \) within a CRN. The equation is:
\[
\frac{d[S]}{dt} = \sum_{\forall \text{rxn} \in \text{CRN}} \kappa(\text{rxn}) \cdot \text{netChange}(S,\text{rxn}) \cdot \prod_{\forall R \in \text{reactants}(\text{rxn})} [R]^{m_{\text{rxn}}(R)}(t)
\]
This can be broken down into its components and translated to code, however, it can also be reformulated to use matrix operations, which in practice are highly optimized:

\[
\frac{d[S]}{dt} = N \cdot v(t)
\]
Where:
\begin{itemize}
    \item \( N \) is the net change matrix, and each element \( N_{ij} \) represents the scaled net change in species \( i \) due to reaction \( j \), including the reaction rate \( \kappa \). It is calculated using the following function:
\begin{minted}{haskell}
let genNetChangeMatrix (crn : CRN) (sl : species List) = 
    Matrix.ofJaggedList (List.foldBack (fun s st -> 
        genNetChangeList crn s :: st) sl [])
\end{minted}
Where \texttt{genNetChangeList} multiplies the reaction speed $c$ with the \texttt{netChange}:
\begin{minted}{haskell}
    let genNetChangeList (crn : CRN) (s : species) = 
        (List.foldBack (fun (Rxn(r,p,c)) st -> 
            c * (calcNetChange (Rxn(r,p,c)) s) :: st) crn [])
\end{minted}
and the \texttt{netChange} is calculated by:
\begin{minted}{haskell}
    let calcNetChange (Rxn(r, p, c)) (s: species) = 
        (getValue p s) - (getValue r s) 

\end{minted}

    \item \( v(t) \) is a vector where each component corresponds to the product of the concentrations of reactants raised to their stoichiometric coefficients for each reaction at time \( t \). This vector is computed by:
\begin{minted}{haskell}
let calcReactionProducts (state : State) (crn : CRN) = 
    vector (
        List.map (fun (Rxn(r, _, _)) ->
            Map.fold (fun s coeff prod -> 
                prod * (state |> Map.find s) ** coeff) 1.0 r
        ) crn
    )
\end{minted}
\end{itemize}

We wanted to further optimize the calculation of \(v(t)\), by using the logarithm to transform it into a dot-product, however, this led to problems due to the possibility of there being zeros.\\

Then a single step can be calculated by finding this derivative for all species:
\begin{minted}{haskell}
    let simulateStepMatrix state crn netChangeMatrix (sl : species List) timestep = 
        let change = calcDerivatives netChangeMatrix 
            (calcReactionProducts state crn) timestep
        List.fold (fun map (s,ds) -> 
            Map.add s ((getValue state s) + ds) map) 
                state (List.zip sl (List.ofArray (Vector.toArray change)))
\end{minted}
And the derivatives comes directly from the discussed formula:
\begin{minted}{haskell}
    let calcDerivatives netChangeMatrix reactionProduct timestep =
        timestep * (netChangeMatrix * reactionProduct) 
\end{minted}

Lastly, the infinite sequence of simulations can then be calculated using an \texttt{unfold}:
\begin{minted}{haskell}
    let simulateReationsMatrix (state : State) (crn : CRN) (timestep : float) = 
    let sl = Set.toList (Set.ofList (
            List.fold
                (fun sp (Rxn(r, p, c)) ->
                    Map.fold (fun keys k _ -> k :: keys) [] r
                    @ Map.fold (fun keys k _ -> k :: keys) [] p
                    @ sp)
                []
                crn
        ))
    let netChangeMatrix = genNetChangeMatrix crn sl

    Seq.unfold (fun st -> let nextState = 
        simulateStepMatrix st crn netChangeMatrix sl timestep
            Some(nextState,nextState)) state

\end{minted}

\subsection{Visualization}
Since this simulation results in a sequence of states, just like the interpreter from the prior section, we can use the visualization code from the interpreter to visualize the simulation.

\subsection{Testing}
To test that the simulations work, simulations on the arithmetic components were tested in the same manner as in the previous section. Due to the slow convergence of the simulations, it is computationally intensive to test the larger programs. Therefore we did not make tests that compare the output from simulated programs to the calculated results. 

\subsection{Assessment}
\section{Compiling CRN++ to chemical reactions}


\section{Evaluation}

This project on CRN++ implementation gave us a hands-on opportunity to deepen our understanding of the F\# programming language while diving into the intriguing world of chemical computation. We successfully enhanced the grammar and integrated matrix operations for more efficient ODE solutions, making our simulations significantly faster. These improvements allowed us to handle larger cases and in combination with our visualization tools, it offered clearer insights into the chemical processes.

Despite these advancements, we encountered challenges, particularly in executing operations within each step correctly to mimic the intended chemical behaviours. This highlighted the importance of proper sequence execution, which we aimed to address through topological sorting in our interpreter — a goal we unfortunately ran out of time to fully implement. 


\newpage
\pagenumbering{roman}
\section{Appendix}
\subsection{Our revised grammar}\label{sec:grammar_revised}

\begin{tabbing}
    $\langle \text{Crn} \rangle$ \,::=\; \= \texttt{'crn=\{'$\langle \text{ConcRootSList} \rangle$'\}'} \\
    
    $\langle \text{ConcRootSList} \rangle$ \,::=\;  $\langle \text{ConcS} \rangle$ ',' $\langle \text{ConcRootList} \rangle$ \\
    \>\textbar \, $\langle \text{RootSList} \rangle$ \\

    $\langle \text{RootSList} \rangle$ \,::=\;  $\langle \text{StepS} \rangle$ \\
    \>\textbar \, $\langle \text{StepS} \rangle$ ',' $\langle \text{RootSList} \rangle$ \\
     
    $\langle \text{ConcS} \rangle$ \,::=\;  \texttt{'conc['$\langle \text{species} \rangle$','$\langle \text{number} \rangle$']'} \\
    
    $\langle \text{StepS} \rangle$ \,::=\;  \texttt{'step['$\langle \text{CommandSList} \rangle$']'} \\
    
    $\langle \text{CommandSList} \rangle$ \,::=\;  $\langle \text{CommandS} \rangle$ \\
    \>\textbar \, $\langle \text{CommandS} \rangle$ \texttt{','} $\langle \text{CommandSList} \rangle$ \\
     
    $\langle \text{CommandS} \rangle$ \,::=\;  \texttt{'rxn['$\langle \text{Expr} \rangle$', '$\langle \text{Expr} \rangle$', '$\langle \text{number} \rangle$']'} \\
    \>\textbar \, \texttt{'ld['$\langle \text{species} \rangle$', '$\langle \text{species} \rangle$']'} \\
    \>\textbar \, \texttt{'add['$\langle \text{species} \rangle$', '$\langle \text{species} \rangle$', '$\langle \text{species} \rangle$']'} \\
    \>\textbar \, \texttt{'sub['$\langle \text{species} \rangle$', '$\langle \text{species} \rangle$', '$\langle \text{species} \rangle$']'} \\
    \>\textbar \, \texttt{'mul['$\langle \text{species} \rangle$', '$\langle \text{species} \rangle$', '$\langle \text{species} \rangle$']'} \\
    \>\textbar \, \texttt{'div['$\langle \text{species} \rangle$', '$\langle \text{species} \rangle$', '$\langle \text{species} \rangle$']'} \\
    \>\textbar \, \texttt{'sqrt['$\langle \text{species} \rangle$', '$\langle \text{species} \rangle$']'} \\
    \>\textbar \, \texttt{'cmp['$\langle \text{species} \rangle$', '$\langle \text{species} \rangle$']'} \\
    \>\textbar \, \texttt{'ifGT['$\langle \text{CommandSList} \rangle$']'} \\
    \>\textbar \, \texttt{'ifGE['$\langle \text{CommandSList} \rangle$']'} \\
    \>\textbar \, \texttt{'ifEQ['$\langle \text{CommandSList} \rangle$']'} \\
    \>\textbar \, \texttt{'ifLT['$\langle \text{CommandSList} \rangle$']'} \\
    \>\textbar \, \texttt{'ifLE['$\langle \text{CommandSList} \rangle$']'} \\
     
    $\langle \text{Expr} \rangle$ \,::=\;  \texttt{$\langle \text{species} \rangle$} \, \{\texttt{'$+$' $\langle \text{species} \rangle$}\}
\end{tabbing}


\subsection{Interpreter plots}\label{sec:interpreter_plots}

\begin{figure}[H]
    \centering
    \includegraphics[width=\textwidth]{report/figures/InterpreterPlots/counterInt.png}
    \caption{Discrete counter}
\end{figure}

\begin{figure}[H]
    \centering
    \includegraphics[width=\textwidth]{report/figures/InterpreterPlots/divisionInt.png}
    \caption{Division}
\end{figure}

\begin{figure}[H]
    \centering
    \includegraphics[width=\textwidth]{report/figures/InterpreterPlots/eulerInt.png}
    \caption{Euler}
\end{figure}

\begin{figure}[H]
    \centering
    \includegraphics[width=\textwidth]{report/figures/InterpreterPlots/factorialInt.png}
    \caption{Factorial}
\end{figure}

\begin{figure}[H]
    \centering
    \includegraphics[width=\textwidth]{report/figures/InterpreterPlots/gcdInt.png}
    \caption{GCD}
\end{figure}

\begin{figure}[H]
    \centering
    \includegraphics[width=\textwidth]{report/figures/InterpreterPlots/piInt.png}
    \caption{Pi}
\end{figure}

\begin{figure}[H]
    \centering
    \includegraphics[width=\textwidth]{report/figures/InterpreterPlots/sqrtInt.png}
    \caption{Sqrt}
\end{figure}

\begin{figure}[H]
    \centering
    \includegraphics[width=\textwidth]{report/figures/InterpreterPlots/sub2Int.png}
    \caption{Sub2}
\end{figure}




\subsection{Compiler plots}\label{sec:compiler_plots}

\begin{figure}[H]
    \centering
    \includegraphics{report/figures/counterWithTimeVars.png}
    \caption{Counter with Time variables}
    \label{fig:counter_time}
\end{figure}

\begin{figure}[H]
    \centering
    \includegraphics[width=\textwidth]{report/figures/SimulatorPlots/counterSim.png}
    \caption{Counter}
\end{figure}

\begin{figure}[H]
    \centering
    \includegraphics[width=\textwidth]{report/figures/SimulatorPlots/divisionSim.png}
    \caption{Division}
\end{figure}

\begin{figure}[H]
    \centering
    \includegraphics[width=\textwidth]{report/figures/SimulatorPlots/eulerSim.png}
    \caption{Euler}
\end{figure}

\begin{figure}[H]
    \centering
    \includegraphics[width=\textwidth]{report/figures/SimulatorPlots/factorialSim.png}
    \caption{Factorial}
\end{figure}

\begin{figure}[H]
    \centering
    \includegraphics[width=\textwidth]{report/figures/SimulatorPlots/gcdSim.png}
    \caption{GCD}
\end{figure}

\begin{figure}[H]
    \centering
    \includegraphics[width=\textwidth]{report/figures/SimulatorPlots/piSim.png}
    \caption{Pi}
\end{figure}

\begin{figure}[H]
    \centering
    \includegraphics[width=\textwidth]{report/figures/SimulatorPlots/sqrtSim.png}
    \caption{Sqrt}
\end{figure}

\begin{figure}[H]
    \centering
    \includegraphics[width=\textwidth]{report/figures/SimulatorPlots/sub2Sim.png}
    \caption{Sub2}
\end{figure}









%\printbibliography

\end{document}
